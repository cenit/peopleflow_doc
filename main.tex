\documentclass{article}
\usepackage[utf8]{inputenc}
\usepackage{listings}
\usepackage{color}
\usepackage[left=2cm,right=2cm,top=2.5cm,bottom=2.5cm]{geometry}
\usepackage{amsmath}

\definecolor{codegreen}{rgb}{0,0.6,0}
\definecolor{codegray}{rgb}{0.5,0.5,0.5}
\definecolor{codepurple}{rgb}{0.58,0,0.82}
\definecolor{backcolour}{rgb}{0.95,0.95,0.92}
 
\lstdefinestyle{mystyle}{
    backgroundcolor=\color{backcolour},   
    commentstyle=\color{codegreen},
    keywordstyle=\color{magenta},
    numberstyle=\tiny\color{codegray},
    stringstyle=\color{codepurple},
    basicstyle=\scriptsize,
    breakatwhitespace=false,         
    breaklines=true,                 
    captionpos=b,                    
    keepspaces=true,                 
    numbersep=5pt,                  
    showspaces=false,                
    showstringspaces=false,
    showtabs=false,                  
    tabsize=2
}

\lstset{style=mystyle}

\def \pf{{\tt pf3 }}

\title{\pf\\a low-res high-density people flow estimator}
\author{G. Guidi, A. Bazzani, S. Sinigardi}
\date{}

\begin{document}
\maketitle

\section{Setup}
Depending on your OS, getting ready to use \pf is different.
First of all, generically we can say that \pf requires {\tt python3}, with {\tt numpy} and {\tt OpenCV} working bindings.
On Ubuntu/WSL, the suggested procedure follows:

%                  python-dev python-numpy libtbb2 libtbb-dev libdc1394-22-dev 
%## fix the "libdc1394 error: Failed to initialize libdc1394"
%$ sudo ln /dev/null /dev/raw1394
\begin{lstlisting}[language=bash, caption=Install on Ubuntu]
$ sudo apt install build-essential cmake git \
                   libgtk2.0-dev pkg-config \
                   libavcodec-dev libavformat-dev \
                   libswscale-dev libjpeg-dev \
                   libpng-dev libtiff-dev \
                   libjasper-dev execstack
$ wget https://repo.continuum.io/archive/Anaconda3-5.0.0.1-Linux-x86_64.sh
$ bash Anaconda3-5.0.0.1-Linux-x86_64.sh
## Remove execution stack flag from OpenSSL lib (better security, required for WSL)
$ execstack -c anaconda3/lib/libcrypto.so.1.0.0
$ conda create --name pf-py36 python=3.6 numpy jupyter
$ source activate pf-py36   # (exit with source deactivate pf-py36)
$ wget https://github.com/opencv/opencv/archive/3.3.0.tar.gz
$ tar zxvf 3.3.0.tar.gz
$ cd opencv-3.3.0
$ mkdir build ; cd build
$ cmake -DCMAKE_BUILD_TYPE=Release \
        -DCMAKE_INSTALL_PREFIX=/usr/local \
        -DWITH_1394=OFF 
        -DPYTHON3_EXECUTABLE=$HOME/anaconda3/envs/pf-py36/bin/python \
        -DPYTHON_INCLUDE_DIR=$HOME/anaconda3/envs/pf-py36/include/python3.6m \
        -DPYTHON_LIBRARY=$HOME/anaconda3/envs/pf-py36/lib/libpython3.6m.so \
        -DPYTHON3_NUMPY_INCLUDE_DIRS=$HOME/anaconda3/envs/pf-py36/lib/python3.6/site-packages/numpy/core/include \
        ..
$ make -j
$ sudo make install
## Remove execution stack flag from OpenCV libs (better security, required for WSL)
$ cd /usr/local/lib
$ find . -type f -name "libopencv_*.so.3.3.0" -exec sudo execstack -c {} \;
$ cd ~/anaconda3/envs/pf-py36/lib/python3.6/site-packages/
$ ln -s /usr/local/lib/python3.6/site-packages/cv2.cpython-36m-x86_64-linux-gnu.so cv2.so
\end{lstlisting}

On macOS, otherwise, the following procedure is known to work, after installing \texttt{Anaconda3} and \texttt{Homebrew}
\begin{lstlisting}[language=bash, caption=Install on macOS]
$ brew install python python3 cmake pkg-config jpeg libpng libtiff openexr eigen tbb numpy
$ brew doctor # be sure to be ready to brew!
$ brew install opencv3 --with-python3 --without-python --with-ffmpeg
$ pip3 install virtualenv virtualenvwrapper
$ pip install virtualenv virtualenvwrapper
$ echo source /usr/local/bin/virtualenvwrapper.sh >> ~/.bash_profile
$ mkvirtualenv pf-py36 -p python3
$ workon pf-py36       # (exit with deactivate pf-py3)
$ pip install numpy
$ cd ~/.virtualenvs/pf-py3/lib/python3.6/site-packages/
$ ln -s /usr/local/opt/opencv3/lib/python3.6/site-packages/cv2.cpython-36m-darwin.so cv2.so
\end{lstlisting}

On Windows (64 bit), we tested this procedure after installing \texttt{Anaconda3}, \texttt{vcpkg} and \texttt{Chocolatey}:
\begin{lstlisting}[language=bash, caption=Install on Windows]
PS \> cinst -y 7zip
PS \> refreshenv
PS \> cd $env:WORKSPACE/vcpkg
PS vcpkg \> .\vcpkg.exe install ffmpeg opencv
PS vcpkg \> cd ..
PS $env:WORKSPACE \> Import-Module BitsTransfer
PS $env:WORKSPACE \> Start-BitsTransfer -Source "https://github.com/opencv/opencv/archive/3.2.0.tar.gz" -Destination "OpenCV.tar.gz"
PS $env:WORKSPACE \> 7z e OpenCV.tar.gz  ; 7z x OpenCV.tar
## open the Anaconda prompt
A \> cd %WORKSPACE%/opencv-3.2.0
A \> mkdir build ; cd build
A \> cmake -G "Visual Studio 15 2017 Win64" "-DCMAKE_BUILD_TYPE=Release" "-DWITH_CUDA=OFF" "-DCMAKE_TOOLCHAIN_FILE=%WORKSPACE%\vcpkg\scripts\buildsystems\vcpkg.cmake" "-DCMAKE_INSTALL_PREFIX=%HOME%" ..
A \> cmake --build . --target install --config Release
A \> conda create --name pf-py36 python=3.6 numpy jupyter
A \> copy %HOME%\Anaconda3\Lib\site-packages\cv2.cp36-win_amd64.pyd %HOME%\Anaconda3\envs\pf-py36\Lib\site-packages\cv2.pyd 
## Finally add to the user's PATH environment variable this folder:
## %HOME%\x64\vc15\bin
\end{lstlisting}

Finally, you should be able to test your installation:
\begin{lstlisting}[language=python, caption=Test]
>>> import cv2
>>> cv2.__version__
'3.2.0'
\end{lstlisting}

After being ready with the prerequisites, we can discuss now what's necessary to run the software

\section{Algorithm}
Let \(A,B,C\) three regions of the same length \(\ell\) (we neglect transverse size, assuming it also constant) and \(n^{\pm}_{A,B,C}\) the number of individuals in the regions moving in opposite directions. 
The individuals move at a velocity \(v\) and are uniformly distributed in the regions (strong assumption, which signifies that our algorithm could be wrong in short times, but should improve with longer periods analyzed).
We denote by \(\Phi^{\pm}\) the incoming flows from left and right in the considered area. In a time interval \(\Delta t\), we call \(n^+_A(t)p\), with \(p \leq 1\), the number of individuals that move from the region A to B. Using similar relations, we have the equations
\begin{align*}
n^+_A(t + \Delta t) &= n^+_A(t)(1 - p) + \Phi^+(t + \Delta t) \Delta t  \\
n^+_B(t + \Delta t) &= n^+_B(t)(1 - p) + n^+_A(t)p                      \\
n^+_C(t + \Delta t) &= n^+_C(t)(1 - p) + n^+_B(t)p                      \\
n^-_A(t + \Delta t) &= n^-_A(t)(1 - p) + n^-_B(t)p                      \\
n^-_B(t + \Delta t) &= n^-_B(t)(1 - p) + n^-_C(t)p                      \\
n^-_C(t + \Delta t) &= n^-_C(t)(1 - p) + \Phi^-(t + \Delta t) \Delta t
\end{align*}

Assuming known (counting pixels different from the background)
\begin{align*}
n_A(t+\Delta t) = n^+_A(t+\Delta t)+n^-_A(t+\Delta t) \\
n_B(t+\Delta t) = n^+_B(t+\Delta t)+n^-_B(t+\Delta t) \\
n_C(t+\Delta t) = n^+_C(t+\Delta t)+n^-_C(t+\Delta t)
\end{align*}

we get a system of nine equation with nine unknown quantities: \(n^{\pm}_{A,B,C}(t + \Delta t)\), \(\Phi^{\pm}(t + \Delta t)\) and \(p\).
The linear system can be solved step by step if we know the initial conditions \(n^{\pm}_{A,B,C}(0)\) and we measure \(n_{A,B,C}(t)\) each time. The probability \(p\) can also be related to the individual velocity
\[
p(t) = \ell \, v(t) \Delta t 
\]
so that in principle we could also obtain an estimate of \(v(t)\) at different times. The error propagation depends from the eigenvalues of the linear system
\begin{align*}
n^+_A(t + \Delta t) +  n^+_A(t)p   - \Phi^+(t + \Delta t)\Delta t &= n^+_A(t)        \\
n^+_B(t + \Delta t) + (n^+_B(t)    - n^+_A(t))p                   &= n^+_B(t)        \\
n^+_C(t + \Delta t) - (n^+_C(t)    - n^+_B(t))p                   &= n^+_C(t)        \\
n^-_A(t + \Delta t) - (n^-_A(t)    - n^-_B(t))p                   &= n^-_A(t)        \\
n^-_B(t + \Delta t) + (n^-_B(t)    - n^-_C(t))p                   &= n^-_B(t)        \\
n^-_C(t + \Delta t) +  n^-_C(t)p   - \Phi^-(t + \Delta t)\Delta t &= n^-_C(t)        \\
n^+_A(t + \Delta t) +  n^-_A(t+\Delta t)                          &= n_A(t+\Delta t) \\
n^+_B(t + \Delta t) +  n^-_B(t+\Delta t)                          &= n_B(t+\Delta t) \\
n^+_C(t + \Delta t) +  n^-_C(t+\Delta t)                          &= n_C(t+\Delta t)
\end{align*}
We expect that the system need to be reset after a characteristic time \(\tau\).
 

\end{document}

